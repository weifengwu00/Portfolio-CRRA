\documentclass{bejournal}
\usepackage{subfigure}
% CDCPrivate: Derivations and verification are available in ../Software/Mathematica/

\input CDCDocStartForBE.tex 

\input HandoutStart.tex

\begin{document}
\input HandoutHeader.tex

\begin{verbatimwrite}{\jobname.title}
Portfolio Choice with CRRA Utility (Merton-Samuelson)
\end{verbatimwrite}
\medskip\medskip\medskip

\input HandoutName.tex

\begin{verbatimwrite}{./body.tex}

  \cite{merton:restat} and \cite{samuelson:portfolio} study the
  optimal portfolio choice of a consumer with Constant Relative Risk
  Aversion utility $\util(c) = (1-\CRRA)^{-1}c^{1-\CRRA}$.  This
  consumer has assets at the end of period $t$ equal to $a_{t}$ and is
  deciding how much to invest in a risky asset with a lognormally
  distributed return factor $\Risky_{t+1}$, $\log \Risky_{t+1} =
  \risky_{t+1} \sim
  \mathcal{N}(\risky-\Evarr/2,\sigma^{2}_{\risky})$,\footnote{\MathFacts~\handoutM{ELogNorm}
    tells us that a variable with this lognormal distribution has an
    expected return factor of
    $\Ex_{t}[e^{\risky_{t+1}}]=e^{\risky}=\Risky$ (where variables
    like $\Risky$ without a subscript are the time-invariant mean).}
  versus a riskfree asset that earns return factor
  $\Rfree=e^{\rfree}$.  Importantly, the consumer is assumed to have
  no labor income and to face no risk except from the investment in
  the risky asset.\footnote{A common interpretation is that this is
    the problem of a retired investor who expects to receive no
    further labor income.  Note however that {\it all} risks other
    than the returns from financial investments have been ruled out;
    for example, health expense risk is not possible in this model,
    though recent research has argued such risk is important later in
    life.}$^{,}$\footnote{Riskless labor income can trivially be added
    to the problem, because its risklessness means that (in the
    absence of liquidity constraints) it is indistinguishable from a
    lump sum of extra current wealth with a value equal to the present
    discounted value (using the riskless rate) of the (riskless)
    future labor income.  Of course, in practice, labor income is not
    riskless, but when labor income is risky the problem no longer has
    the tidy analytical solution we present here and must be solved
    numerically.  See \cite{SolvingMicroDSOPs} for an introduction to
    numerical solution methods.}

Both papers consider a multiperiod optimization problem, but here we
examine the problem of a consumer in a period $t$ which is the
second-to-last period of life (the insights, and even the formulas,
carry over to the multiperiod case).

If the period-$t$ consumer invests proportion $\riskyshare$ in the risky asset, 
spending all available resources in the last period of life $t+1$ will yield:
\begin{eqnarray}
        c_{t+1} & = & \left(\Rfree(1-\riskyshare)+\Risky_{t+1}\riskyshare\right)a_{t} \notag
\\ & = & \underbrace{\left(\Rfree+(\Risky_{t+1}-\Rfree)\riskyshare\right)}_{\equiv \Rport_{t+1}}a_{t} \label{eq:RportDef}
\end{eqnarray}
where $\Rport_{t+1}$ is the realized arithmetic\footnote{Google ``arithmetic geometric mean wiki'' for a refresher on the difference between arithmetic and geometric means.} return factor for the portfolio.

For mathematical analysis (especially under the assumption of CRRA
utility) it would be convenient if we could approximate the realized
arithmetic portfolio return factor by the realized geometric return
factor
$\Rport_{t+1}=\Rfree^{1-\riskyshare}\Risky_{t+1}^{\riskyshare}$,
because then the logarithm of the return factor would be $\rport_{t+1}
= \rfree (1-\riskyshare)+\risky_{t+1}\riskyshare = \rfree +
(\risky_{t+1}-\rfree)\riskyshare = \rfree + \eprem_{t+1} \riskyshare$
and the realized `portfolio excess return' would be simply
$\rport_{t+1}-\rfree = \riskyshare\eprem_{t+1}$.  Unfortunately, for
$\riskyshare$ values well away from 0 and 1 (that is, for any {\it
  interesting} values of portfolio shares), the geometric mean is a
badly biased approximation to the arithmetic mean when the variance of
the risky asset is substantial.

\cite{cvAppendix} point out that a much better approximation is obtained by assuming
%\begin{eqnarray}
%  \rport_{t+1} &  = & \rfree + \riskyshare (\eprem_{t+1}+\sigma^{2}_{\risky}/2) - \riskyshare^{2}\sigma^{2}_{\risky}/2
%\end{eqnarray}
\begin{eqnarray}
  \rport_{t+1} & = & \rfree+ \riskyshare \eprem_{t+1}+\riskyshare \sigma^{2}_{\risky}/2 - \riskyshare^{2}\sigma^{2}_{\risky}/2 \label{eq:rportCV}.
\end{eqnarray}

To see one virtue of this approximation, note (using \handoutM{NormTimes} and \handoutM{SumNormsIsNorm}) that since the mean and variance of $\eprem_{t+1} \riskyshare$ are respectively $\riskyshare(\risky - \sigma^{2}_{\risky}/2- \rfree)$ and $\riskyshare^{2} \sigma^{2}_{\risky}$, \handoutM{LogELogNormTimes} implies that  
\begin{eqnarray}
%   (\risky_{t+1}-\sigma^{2}_{\risky}/2)\riskyshare & \sim & \mathcal{N}(\riskyshare (\risky-\sigma^{2}_{\risky}/2),\riskyshare^{2} \sigma^{2}_{\risky}) \\
 \log \Ex_{t}[e^{\eprem_{t+1} \riskyshare}] & = & \riskyshare (\risky - \rfree - \sigma^{2}_{\risky}/2) + \riskyshare^{2}\sigma^{2}_{\risky}/2 
\end{eqnarray}
%so 
%\begin{eqnarray}
%  \log \Ex_{t}[e^{\riskyshare (\risky_{t+1}-\sigma^{2}_{\risky}/2)}] & = & \riskyshare (\risky-\sigma^{2}_{\risky}/2)+\riskyshare^{2}\sigma^{2}_{\risky}/2
%\end{eqnarray}
which means that exponentiating then taking the expectation then taking the logarithm of \eqref{eq:rportCV} gives 
\begin{eqnarray}
   \log \Ex_{t}[e^{\rport_{t+1}}] & = & \log e^{\rfree} + \log \Ex_{t}[e^{\riskyshare \eprem_{t+1}}] + \log e^{\riskyshare\sigma^{2}_{\risky}/2 - \riskyshare^{2}\sigma^{2}_{\risky}/2}
\\ & = & 
\rfree + \riskyshare (\risky-\rfree-\sigma^{2}_{\risky}/2)+\riskyshare^{2}\sigma^{2}_{\risky}/2 +\riskyshare \sigma^{2}_{\risky}/2 - \riskyshare^{2}\sigma^{2}_{\risky}/2 
\\ \log \Ex_{t}[e^{\rport_{t+1}}] - \rfree & = &  \riskyshare (\risky-\rfree) \label{eq:rportDist}
\end{eqnarray}
or, in words: The expected excess portfolio return is equal to the proportion invested in the risky asset times the expected return of the risky asset.\footnote{We use the word `return' always to mean the logarithm of the corresponding `factor'; and when not explicitly specified, we always take the expectation before taking the log; if we wanted to refer to $\Ex_{t}[\rport_{t+1}]$ we would call it the expected log portfolio return (to distinguish it from the expected portfolio return $\log \Ex_{t}[e^{\rport_{t+1}}]$.}

Under these assumptions, the expectation as of date $t$ of utility at date $t+1$
is:
\begin{eqnarray}
  \Ex_{t}[\util(c_{t+1})] & \approx & (1-\CRRA)^{-1}\Ex_{t}\left[\left(a_{t}e^{\rfree}e^{\riskyshare \eprem_{t+1}+\riskyshare(1-\riskyshare)\sigma^{2}_{\risky}/2 }\right)^{1-\CRRA}\right] \notag
\\                      & \approx & (1-\CRRA)^{-1}\Ex_{t}\left[(a_{t}\Rfree)^{1-\CRRA}\left( e^{\riskyshare \eprem_{t+1}+\riskyshare(1-\riskyshare)\sigma^{2}_{\risky}/2 }\right)^{1-\CRRA}\right] \notag
\\                      & \approx & (1-\CRRA)^{-1}(a_{t}\Rfree)^{1-\CRRA}\Ex_{t}\left[e^{(\riskyshare \eprem_{t+1}+\riskyshare(1-\riskyshare)\sigma^{2}_{\risky}/2)  (1-\CRRA)}\right] \notag
\\                      & \approx & \underbrace{(1-\CRRA)^{-1}(a_{t}\Rfree)^{1-\CRRA}}_{\text{constant $< 0$}}\underbrace{e^{ (1-\CRRA)\riskyshare(1-\riskyshare)\sigma^{2}_{\risky}/2}\Ex_{t}\left[e^{\riskyshare \eprem_{t+1}  (1-\CRRA)}\right]}_{\text{excess return utility factor}}
 \label{eq:exputil}
  \end{eqnarray}
where the first term is a negative constant under the usual assumption that relative risk aversion $\CRRA>1.$

  Our foregoing assumptions imply that $\riskyshare (1-\CRRA) \eprem_{t+1}
   \sim \mathcal{N}(\riskyshare (1-\CRRA)(\eprem -
  \Evarr/2),(\riskyshare(1-\CRRA))^{2}\Evarr)$ (again using 
  \handoutM{LogELogNormTimes}).  With a couple of extra
  lines of derivation we can show that the log of the expectation in \eqref{eq:exputil} is
\begin{eqnarray}
  \log \Ex_{t}\left[e^{\riskyshare \eprem_{t+1}  (1-\CRRA)}\right] & = & {(1-\CRRA)\riskyshare \eprem-(1-\CRRA)\riskyshare\Evarr/2+ ((1-\CRRA)\riskyshare)^{2}\Evarr/2} \notag
\\  & = & {(1-\CRRA)\riskyshare \eprem-(1-\CRRA)\riskyshare(1-\riskyshare(1-\CRRA))\Evarr/2} \notag
\\  & = & {(1-\CRRA)\riskyshare \eprem-(1-\CRRA)\riskyshare(1-\riskyshare)\Evarr/2-\CRRA (1-\CRRA)\riskyshare^{2}\Evarr/2}. \label{eq:Ex}
\end{eqnarray}

Substitute from \eqref{eq:Ex} for the log of the expectation in
\eqref{eq:exputil} and note that the resulting expression simplifies because it contains
${(1-\CRRA)\riskyshare\Evarr/2-(1-\CRRA)\riskyshare\Evarr/2}=0$; 
thus the log of the `excess return utility factor' in \eqref{eq:exputil} is
\begin{equation}
  -(\CRRA-1)\riskyshare \eprem - (\CRRA-1)(- \CRRA \riskyshare^{2}\Evarr/2)
\end{equation}
and the $\riskyshare$ that minimizes the log will also
minimize the level; minimizing this when $\CRRA>1$ is equivalent to
maximizing the terms multiplied by $-(\CRRA-1)$, so our problem
reduces to
\[
\max_{\riskyshare}~~ \riskyshare \eprem -\CRRA\riskyshare^{2}\Evarr/2 
\]
with FOC
\begin{eqnarray}
         \eprem-\riskyshare\CRRA\Evarr  & = & 0  \notag \\ 
\riskyshare & = & \left(\frac{\eprem}{\CRRA \Evarr}\right)
. \label{eq:riskyshareMS}
\end{eqnarray}

Equation \eqref{eq:riskyshareMS} says\footnote{This expression differs
  slightly from that derived by \cite{cvAppendix}, because we adjust
  the mean logarithmic return of the risky investment for its variance
  in order to keep the mean return factor constant, which makes
  comparisons of alternative levels of risk more transparent.}  that the
consumer allocates more of his portfolio to the high-risk, high-return
asset when
\begin{enumerate}
\item the amount $\eprem$ by which the risky
asset's return exceeds the riskless return is greater
\item the consumer is less risk averse ($\CRRA$ is lower)
\item riskiness $\sigma^{2}_{\risky}$ is less
\end{enumerate}
If there is no excess return, nothing will be put in the risky asset.  Similarly, if 
risk aversion or the variance of the risk is infinity, again nothing 
will be put in the risky asset.\footnote{See the appendix for a figure
  showing the quality of the approximation.}

%\begin{comment} % Removed because the calibration yields yields for \CRRA = 1 a portfolio share of 2 according to the approximate formula but the exact formula implies the portfolio share cannot reach 1; for a better calibration, the point is less compelling 
This formula hints at the existence of an `equity premium puzzle'
(\cite{mehra&prescott:puzzle}).  Interpreting the risky asset as the
aggregate stock market, the annual standard deviation of the log of
U.S.\ stock returns has historically been about $\sigma_{\risky}=0.2$
yielding $\Evarr = 0.04$.  The equity premium over historical periods
has been something like $\eprem = 0.08$ (eight percent).  With risk
aversion of $\CRRA=2$ this formula implies that the share of risky
assets in your portfolio should be $0.08/0.08$ or 100 percent!  The
fact that most people have less than 100 percent of their wealth
invested in stocks is the `stockholding puzzle,' the microeconomic
manifestation of the equity premium puzzle
(\cite{haliassos&bertaut:fewholdstocks}).

To avoid the problems caused by a prediction of a risky portfolio
share greater than one, we can calibrate the model with more modest
expectations for the equity premium.  Some researchers have argued
that when evidence for other countries and longer time periods is
taken into account, a plausible average value of the premium might be
as low as three percent.  Figure~\ref{fig:Port} shows the relationship
between the portfolio share and relative risk aversion for a
calibration that assumes a modest premium of 3 percent and a large
standard deviation of $\sigma=0.2$.  Even when risks are this high and
the premium is this low, if relative risk aversion is close to
logarithmic ($\CRRA = 1$ the investor wants to put well over half of
the portfolio in the risky asset.  Only for values of risk aversion
greater than 2 does the predicted portfolio share reach plausibly
small values.

%\end{comment}


A final interesting question is what the expected rate of return on
the consumer's portfolio will be once the portfolio share in risky
assets has been chosen optimally.  Note first that \eqref{eq:rportDist}
implies that 
\begin{eqnarray}
  \log \Ex_{t}[e^{\rport_{t+1}-\rfree}] & = & \riskyshare \eprem  \label{eq:eport}
\end{eqnarray}
while the variance of the log of the excess return factor for the portfolio is $\sigma^{2}_{\rport} = \riskyshare^{2} \sigma^{2}_{\risky}.$

Substituting the solution \eqref{eq:riskyshareMS} for $\riskyshare$ into \eqref{eq:eport}, we have
\begin{eqnarray}
  \riskyshare \eprem & = & \left(\frac{\eprem^{2}}{\CRRA \Evarr}\right)  \notag
\\ & = &  (\eprem/\sigma_{\risky})^{2}/\CRRA \label{eq:rportPremOpt}
\end{eqnarray}
which is an interesting formula for the excess return of the optimally
chosen portfolio because the object $\eprem/\sigma_{\risky}$ (the
excess return divided by the standard deviation) is a well-known tool
in finance for evaluating the tradeoff between risk and return (the
`Sharpe ratio').  Equation \eqref{eq:rportPremOpt} says that the consumer will
choose a portfolio that earns an excess return that is directly
related to the (square of the) Sharpe ratio and inversely related to the risk aversion
coefficient.  Higher reward (per unit of risk) convinces the consumer
to take the risk necessary to earn higher returns; but higher risk
aversion convinces the investor to sacrifice return for safety.

Finally, we can ask what effect an exogenous increase in the risk of the 
risky asset has on the endogenous riskiness of the portfolio once the consumer
has chosen optimally.  The answer is surprising: The variance of the optimally-chosen
portfolio is 
\begin{eqnarray}
\riskyshare^{2} \sigma^{2}_{\risky} & = & \left(\frac{\eprem}{\CRRA \Evarr}\right)^{2} \sigma^{2}_{\risky}
\\ & = & \left(\frac{(\eprem/\CRRA)^{2}}{\Evarr}\right)
\end{eqnarray}
which is actually {\it smaller} when $\sigma^{2}_{\risky}$ is larger.
Upon reflection, maybe this makes sense.  Imagine that the consumer
had adjusted his portfolio share in the risky asset downward just
enough to return the portfolio's riskiness to its original level that
it had before the increase in risk.  The consumer would now be bearing
the same degree of risk but for a lower (mean) rate of return (because
of his partial reduction in exposure to the risky asset).  It makes
intuitive sense that the consumer will not be satisfied with this
``same riskiness, lower return'' outcome and therefore that the
undesirableness of the risky asset must have increased enough to make
him want to hold even less than the amount that would return his
portfolio's riskiness to its original value.


\begin{figure}[h]
\caption{The Risky Portfolio Share $\riskyshare$ and Relative Risk Aversion $\CRRA$} \label{fig:Port}\centering
\subfigure[The Approximate Risky Portfolio Share $\riskyshare$ Declines as Relative Risk Aversion $\CRRA$ Increases]{
    \label{fig:Port:a}
    \fbox{\CDCFig{ShareVsCRRA}}
}\\
\vspace{.1in} \subfigure[The Approximation Error for the Portfolio Share in Risky Assets $\riskyshare$ Is Small] {
    \label{fig:Port:b}
    \fbox{\CDCFig{ShareApproxErr}}
} \begin{flushleft} \footnotesize Note: The approximation error is computed by solving for the exactly optimal
portfolio share numerically.  See the \texttt{Portfolio-CRRA-Derivations.nb} Mathematica notebook for details.
\end{flushleft}
\end{figure}

\end{verbatimwrite}
\input ./body.tex


\input bibMake

\end{document}


